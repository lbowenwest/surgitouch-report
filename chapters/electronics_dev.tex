%!TEX root = ../report.tex

\section{Motors}\label{sec:motors}

\subsection{Motor Selection}\label{sub:motor-selection}

% ** Louisa writing a bit in the mechanism development section** -
% inserted some relevant stuff into here will sort out later

% \emph{The client requirements state that 20 steps in torque should be
% set as a target. Research into various motors has shown that brushless
% motors provide lower inertia compared to those with brushes, however for
% the same torque and power brushless motors are around double the price
% in the range of £300 upward per motor. The controllers required by
% brushless motors are also distinctly more expensive than those required
% for brushed motors. Hence with budget limitations a compromise was made
% with the client to use two motors provided by the mechatronics in
% medicine department with a rated nominal continuous torque of 45.5 mNm
% and a stall torque which can go up to 212mNm (short term operation range
% only -- danger of overheating) which is sufficient for this application,
% the torque range will just be reduced to around 65\% of the preliminary
% test values. The retail value of these motors is \textbf{£152.96 each.}}

% \emph{Taking into account the motor's peak torque providing capabilities
% if the stall torque of 212mNm is reached then the motor stalls and
% effectively the whole mechanism shuts down.}

Using preliminary test results and PDS what is our required torque,
continuous torque, speed torque gradient graph, stall torque, stall
current, operating range, radial play, nominal voltage, power, rotor
inertia, efficiency, maximum dimensions and mass.

Thermal data. Thermal resistance.

Servomotor, brushless (high inertia and can feel it), brushed motor.

However due to budget limitations a motor with half the torque was all
that was practical, but this leaves an area for future
improvement/investment. {[}Future more accurate testing with final
prototype may prove otherwise{]}

Our motor data

A-max 32 Ø32 mm, Graphite Brushes, 20 Watt, with terminals

High Power Part number 236669

\begin{longtable}[]{@{}ll@{}}
  \toprule
  Nominal voltage & 24 V\tabularnewline
  \midrule
  \endhead
  Nominal torque (max. continuous torque) & 45.5 mNm\tabularnewline
  Stall torque & 212 mNm\tabularnewline
  Stall current & 6.02 A\tabularnewline
  Rotor inertia & 45.3 gcm²\tabularnewline
  Weight & 240 g\tabularnewline
  \bottomrule
\end{longtable}

\subsection{Motor Characterisation}\label{sub:motor-characterisation}

The motors supplied by our Assistant Supervisor had already been used,
so their voltage-current characteristics will have changed compared to
their data sheet values. Therefore, each motor was tested by setting the
voltage using the PWM levels (0-127) and reading the current values
using the Roboclaw's inbuilt current sensors. The maximum current was
set at 1.3 Amps to prevent the motors from overheating

The data was then plotted to determine the motor's current threshold and
armature resistance.

Care was taken to ensure that temperature couldn't affect the readings
by turning the motors off between each reading.

% \subparagraph{\texorpdfstring{6.1.2.1 Motor 1}{ 6.1.2.1 Motor 1}}\label{motor-1}
\paragraph{Motor 1}\label{par:motor-1}

% \includegraphics[width=5.24419in,height=3.94271in]{media/image21.png}

Threshold current: 0.8 Amps

Armature resistance: 5.71 Ohms

% \subparagraph{6.1.2.2 Motor 2}\label{motor-2}
\paragraph{Motor 2}\label{par:motor-2}

% \subparagraph{\texorpdfstring{\protect\includegraphics[width=5.33588in,height=4.00868in]{media/image19.png}}{}}\label{section}

Threshold current: 0.7 Amps

Armature resistance: 4.39 Ohms

\section{Controller Selection }\label{sec:controller-selection}

The primary consideration for the controller selection was the desire to
have as little lag as possible. It was also important that the
controller be dual axis, and to synchronise between the two axis. Also,
if possible, current or torque control would be beneficial. While
searching for suitable controllers, the majority of the high end
candidates were out of the range of our budget
% ({[}\href{http://www.maxonmotor.co.uk/maxon/view/product/control/Positionierung/546714}{\emph{http://www.maxonmotor.co.uk/maxon/view/product/control/Positionierung/546714}}{]}
costing approximately £600 for dual axis, leaving no leeway for purchase
of motors, manufacturing costs, and other required components). For the
purpose of getting a working prototype it was decided to purchase a
simpler controller with dual axis support for experimentation and
development, with the possibility of upgrading controllers and motors to
get a quicker, more accurate response at a later date for the final
design. The controller selected, whilst it did not have the ability to
control the current directly, was able to read the currents to each
motor and relay this information back over a serial connection to a
microcontroller.

Additionally, the Roboclaw sets the voltage using PWM by splitting the
voltage range (0-24V) into 7 bits (0-127), meaning it only increments
the voltage applied by 0.189 V.

Single axis control / dual axis control (no lag and automatically
synchronized), price comparison. Possibility of producing a PCB (not
required because benefits size/speed-wise are not necessary for given
specifications), however could be incorporated in the long term to
improve prototype. Explain arduino option was easier and allowed focus
to go on mechanical design optimisation.

\section{Software Choice}\label{sec:software-choice}

Single axis control / dual axis control (no lag and automatically
synchronized), price comparison. Possibility of producing a PCB (not
required because benefits size/speed-wise are not necessary for given
specifications), however could be incorporated in the long term to
improve prototype. Explain arduino option was easier and allowed focus
to go on mechanical design optimisation.

\section{System modelling}\label{sec:system-modelling}

\subsubsection{Governing equations}\label{governing-equations}

In order for our joystick to apply the correct amount of torque given
its position, a form of torque control must be applied

To start off, we consider the simple DC motor circuit in figure 1 with
its corresponding electrical and mechanical equations:

% \includegraphics[width=6.50000in,height=1.75000in]{media/image16.png}

From equation 3, it can be seen that the Torque of the motor is directly
proportional to the current it draws, with the torque constant being the
proportionality constant. Thus, the torque which the motors apply can be
controlled by varying the current supplied.

In its final application, it is assumed that the surgeon will be moving
the joystick through small angles at negligible angular velocities,
resulting in a quasi-static version of the model. This allows for a
great simplification by setting omega dot = omega = 0 .

Additionally, referring to the specification sheet for the Maxon A-max
32 motor (link to motor selection section) we will be using, we see that
the inductance of the motor is 0.556 mH, which compared to its armature
resistance of 5.7 ohms is negligible and thus allows for a second
simplification: L = 0.

Thus, the motors transfer function can be represented by Ohm's Law with
the addition of the torque constant equation, resulting in a system with
the voltage being the input and the torque the output:

% \includegraphics[width=6.50000in,height=1.27778in]{media/image23.png}

To model the entire joystick system as a block diagram, we set the
position as the input and the torque as the output.

As the Roboclaw controller can only set the voltage supplied to the
motors using Pulse Width Modulation (PWM), to obtain our desired current
we multiply the location dependent current equation obtained in section
X by the armature resistance R.

Representing the system as a block diagram:

% \includegraphics[width=6.50000in,height=1.54167in]{media/image24.png}

Due to the negligible inductance of the motors, the system is a zeroth
order system and therefore doesn't require a feedback loop.

This can be accounted for by simply changing the resistance value in the
block diagram, and subtracting 0.8 Amps from the desired current i\_d ,
as follows:

% \includegraphics[width=6.50000in,height=1.61111in]{media/image15.png}

\subsubsection{Temperature Effects}\label{temperature-effects}

Given the nature of their final application, the motors will constantly
be at their stall torque and current, resulting in significant
temperature increases.

Using the temperature coefficient into the equation for the temperature
dependence of the resistance, we find that the resistance at maximum
operating temperature (125 °C) is 42\% greater than that at ambient
temperature (20 °C).

Alternatively, as our preliminary test results showed that a 10\%
increase in force is noticeable, a temperature rise of 25 °C would
result in a noticeable change on the surgeon's finger.

Two possible solutions were explored in order to mitigate the effects of
the temperature variations, the first being PID current control and the
second using a dynamic resistance which was constantly updated in the
code.

\section{Current control designs}\label{sec:current-control-designs}

\subsection{PID design}\label{sub:pid-design}

This variation of operating temperature depends greatly on the operating
conditions of the motors, and cannot be modelled using basic block
diagrams. Thus, to design a suitable PID controller for this use, a
model of the system was created in SimuLink.

% \includegraphics[width=6.50000in,height=2.95833in]{media/image14.png}

% \includegraphics[width=6.50000in,height=2.02778in]{media/image18.png}

This model was designed to reflect real life operating conditions as
accurately as possible. The desired current is amplified by the real
winding resistance (5.7 ohms) to set the required voltage. A PWM
generator is then used to power the DC motor via an H-Bridge, with the
DC Voltage Source representing the unknown voltage source producing 0.8
Amps. The motor shaft is held steady with an ideal angular velocity
source providing 0 rad/s (to model the surgeon resisting the motion) and
a torque counteracting that of the motor for the given current.

Furthermore, a constant temperature increase of 1 °C/s is applied to the
motor to model its temperature dependence.

% \includegraphics[width=6.50000in,height=2.27778in]{media/image17.png}

Here we see that the current reaches 1 Amp immediately, but drops by 5\%
after 10 seconds of temperature increase.

A PID controller was then placed in the Simulink model, by using the
desired current as the set point, the actual current as the input and
the resulting error signal as the output. In the real model, we will use
a software PID in the Arduino Leonardo, using the RoboClaw's inbuilt
current sensors to measure the actual current.

% \includegraphics[width=6.50000in,height=2.98611in]{media/image22.png}

Simulinks' inbuilt PID tuner was used to find the ideal $K_p$, $K_i$ and
$K_d$ values, subject to the PDS criteria:

- 200 ms response time (rise time)

- 0\% overshoot

- 0\% steady state error

\begin{longtable}[]{@{}ll@{}}
  \toprule
  $K_p$ & 0.003\tabularnewline
  $K_i$ & 24\tabularnewline
  $K_d$ & 0\tabularnewline
  \bottomrule
\end{longtable}

Using this controller in the model, we get the following response:

% \includegraphics[width=6.50000in,height=2.27778in]{media/image20.png}

With a settling time of 0.166 seconds, this response fulfils all our
design criteria.

Additionally, the system has ideal disturbance rejection if it were
exposed to any sudden operating condition changes.

To give an example, the motors were exposed to a sudden temperature
increase of 30 °C after 3 seconds of operation:

% \includegraphics[width=6.50000in,height=2.27778in]{media/image8.png}

\subsection{Dynamic Resistance}\label{sub:dynamic-resistance}

This fairly straightforward method for accounting for the temperature
dependence of the windings resistance simply evaluates the resistance
using the current sensors as follows:

A desired current is calculated based on the location of the joystick.
The motors current-voltage characteristic graph (found in section X) is
then used to apply the corresponding voltage using the original
resistance.

The current sensors are then used to read in the actual current values
in the circuit, and the new resistance is obtained as follows:

\(R\  = \ \frac{V}{I_{actual,\ average} - I_{\text{threshold}}}\)

This resistance is then used to apply the correct voltage to obtain the
desired current.

\subsection{Conclusion}\label{sub:conclusion}

Both methods were implemented into the code of our joystick and tested.
Using the simulated gain values for the PID determined above, it was
observed that the joystick would oscillate a significant amount when a
force was being applied. K\_p, K\_i and K\_d values were altered and
tested, but the joystick would still jitter as soon as it was trying to
apply a torque.

One of the reasons for this jitteriness is due to the fact that, based
on the joystick position, the desired current is at a level which the
system cannot supply based on the current resolution of 0.03 Amps.
Therefore, the PID is trying to enforce a current which is physically
impossible. To account for this, the PID was redesigned to control the
PWM logic levels (0-127), however the unstableness was still observed.

The dynamic resistance method showed a stronger and more stable response
in comparison, and will thus be implemented in the final version of the
joystick. The current resolution was also accounted for in this method,
by only letting the desired current increment in 0.03 Amp steps. The
tests to evaluate its accuracy can be found in section X.

\section{Cooling System}\label{sec:cooling-system}

Didn't have a fucking clue where to put this. Plan on doing some simple
fan calcs.\\
Our fans are capable of 9.35m\^{}3 / hr. How often is the air inside the
casing replaced? Can we then set this up into a simple heat transfer
equation to see how quickly the motor would cool from its max operating
temp?

Need to take note that this uses advanced CFD to actually work out, but
we can probably get some ``that'll do donkey'' type calculations.

\section{Power Supply}\label{sec:power-supply}

What we're allowing for\ldots{} two 0.6W fans, arduino, roboclaw, two
motors, losses in wiring?

Potentiometer? Switch losses?
