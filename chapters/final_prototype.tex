\section{Gimbal Mechanism}
\label{secgimbal_mechanism}

Design aspects of every part we made/used and how they all interact.

Minimum allowable sizes (stress analysis – max forces applied). Also maximum sizes due to lack of fit / clashes of moving mechanism.

Friction Analysis and low intertia

% section gimbal_mechanism (end)

\section{Electrical Circuitboards}
\label{secelectrical_circuitboards}

Picture of boards with diagrammatic explanations

% section electrical_circuitboards (end)

\section{Haptic Feedback Implementation (Software)}
\label{sechaptic_feedback_implementation_}

Flow Charts

% section haptic_feedback_implementation_ (end)

\section{Cooling Considerations}
\label{seccooling_considerations}
Simply put two fans and holes in sidewalls of main casing to keep all components (motor, controller and arduino cool).
Limit device use time, so that motors do not over heat. For longer usage times a more expensive and efficient cooler is required because the motors heat up to a large extent.

Motors left in open air and the plastic insulated parts are attached to the main housing via screws. A slotted base support is used to hold weight of the motor and reduce stresses at the main mechanism housing (although stresses are small already, a good idea to add support to reduce even more, to point where screws simply orient motor in correct direction and do not support the weight of it - in case of vibrations of motor and repeated cyclic loading causing fatigue failure of housing due to cyclic bending stresses).

% section cooling_considerations (end)

\section{Casing and Ergonomic Design}
\label{seccasing_and_ergonomic_design}

% section casing_and_ergonomic_design (end)

\section{Placement of Electronic Components}
\label{secplacement_of_electronic_components}

Map out with dimensions the base of housing. Show different coloured wires to/from each electronic component. Consider different ways of tidying wires and choose an option.

%section placement_of_electronic_components (end)